%%%%%%%%%%% bismi al-ll_ah ar-ra.hmAni ar-ra.hIm %%%%%%%%%

\documentclass[dvipsnames]{beamer}

\usepackage[utf8]{inputenc}
\usepackage{beamerthemesplit}
\usetheme{Warsaw}
%\usecolortheme{default}
\usepackage[]{natbib}
\usepackage{graphicx}
\usepackage{hyperref}
\usepackage{multimedia}
\usepackage[spanish]{babel}

\usepackage{enumerate}
\usepackage{amsmath,amsthm}
\usepackage{amsfonts,amssymb,latexsym}
\usepackage{keystroke}

\usepackage{relsize}

\title[Presentación de la Ponencia]{
  Uso de \htmladdnormallink{\texttt{Sagemath}}{https://www.sagemath.org/}\\ 
  en la\\
  Investigación y la Docencia}
\author
[Jornadas de la OSL de la UGR]
{\htmladdnormallink{Fco. M. García Olmedo}{https://github.com/ringstellung/sage_ponencia} y P. González Rodelas}
\institute{\textsc{Dpto. Álgebra y Mat. Aplicada}\\[5mm]
  \htmladdnormallink{\includegraphics[width=0.15\textwidth]{./sage_logo_new_l_hc_edgy-nq8.png}}
      {https://www.sagemath.org/}
}
\date{28 de septiembre de 2018} 
\AtBeginSection{\frame{\frametitle{Tabla de contenidos}
\tableofcontents[currentsection]}}

\begin{document}

\frame{\titlepage}
\frame{\tableofcontents}

\section{Descripción General}

\begin{frame}[fragile]{¿Qué es \texttt{sage}?}
  \frametitle{¿Qué es \texttt{Sagemath}?}
  \begin{block}{Clave: Es un sistema computacional de cálculo escrito
      en \texttt{Cython}%, una bifurcación de \texttt{Pyrex}
    }
    Reune y \textcolor{red}{unifica} bajo un solo techo, lenguaje y
    gerarquía de objetos toda una colección de software matemático,
    \textcolor{red}{rellenando} los huecos de funcionalidad de todos
    ellos.
  \end{block}\pause 
  \begin{itemize}[<+->]
  \item Previamente se llamó \textcolor{red}{\texttt{Sage}}, acrónimo
    de \textcolor{OliveGreen}{``System for Algebra and Geometry
      Experimentation''}.
  \item El proyecto está liderado por \textcolor{red}{William Stein}
    de la Univ. de Washington.
  \item Crear una \textcolor{red}{alternativa} a los
    \textcolor{red}{sistemas propietario} llevaría sería un
    \textcolor{red}{trabajo colosal}.
  \end{itemize}
\end{frame}

\begin{frame}[fragile]{Características}
  \frametitle{¿Qué es \texttt{Sagemath}?}
  \begin{itemize}[<+->]
  \item Existencia de gran cantidad de \textcolor{red}{software bien
      probado} pero de naturaleza heterogénea, que
    \textcolor{red}{podría ser unificado}.
  \item \textcolor{red}{Desarrollado por}: estudiantes, becarios y
    profesionales.
  \item Financiado con trabajo \textcolor{red}{voluntario} y
    \textcolor{red}{donaciones}.
  \item Distribuido bajo licencia \textcolor{red}{GNU V2 o posterior}.
%  \item Se puede usar en la \textcolor{red}{versión en línea}
%    \htmladdnormallink{\texttt{cocalc.com}}{https://cocalc.com/}.
  \end{itemize}
\end{frame}

\begin{frame}[fragile]{Características}
  \frametitle{Características}
  \begin{itemize}[<+->]
  \item Pretende \textcolor{red}{rendir como} \texttt{C} con un código
    escrito en \texttt{Python}.
  \item El \textcolor{red}{lenguaje} de programación es
    \textcolor{red}{actualmente} \texttt{Python 2.7}, con toda su
    funcionalidad.
  \item Puede ejecutarse en una \textcolor{red}{consola} basada en
    \texttt{iPython}.
  \item Incluye una \textcolor{red}{interfax} gráfica (servidor SageNB
    o Jupyter).
  \item \textcolor{red}{Cálculo simbólico} usando \texttt{Maxima} y \texttt{SymPy}.
  \item \textcolor{red}{Bibliotecas propias} especiales, entre otros
    para Teoría de Números.
  \item Herramientas para el \textcolor{red}{proceso de imágenes}
    mediante \texttt{Pylab} y \texttt{Python}.
  \end{itemize}
\end{frame}

\begin{frame}[fragile]{Características}
  \frametitle{Características}
  \begin{itemize}[<+->]
  \item Capacidad de \textcolor{red}{importar} y
    \textcolor{red}{exportar} datos, imágenes, vídeos, sonido, etc.
  \item Capacidad de \textcolor{red}{embeber} \texttt{Sagemaht} en
    documentos \textcolor{red}{\LaTeX}.
  \end{itemize}
\end{frame}

\section{Aspectos Técnicos}

\begin{frame}[fragile]{Peculiaridades}
  \frametitle{Peculiaridades}
  \begin{block}{Clave: \texttt{Sagemath} soporta cálculos con objetos
      de diferentes sistemas algebraicos de computación ``bajo un
      mismo techo''}
    éstos son \textcolor{red}{\texttt{GP/Pari}},
    \textcolor{red}{\texttt{GAP}}, \textcolor{red}{\texttt{Singular}}
    y \textcolor{red}{\texttt{Maxima}}, usando una interfax común y un
    lenguaje de programación límpio.
  \end{block}\pause 
  \begin{itemize}[<+->]
  \item Los archivos de \texttt{Sagemath} tienen extensión
    \textcolor{red}{\texttt{.sage}} y pueden ser leídos con distintas
    funcionalidades: \textcolor{red}{\texttt{load}} (\textcolor{OliveGreen}{estático}) o
    \textcolor{red}{\texttt{attach}} (\textcolor{OliveGreen}{dinámico}).
  \item Una vez \textcolor{red}{leídos} son
    \textcolor{red}{traducidos} a \texttt{Python} e
    \textcolor{red}{interpretados}.
  \item Es posible dar \textcolor{red}{rapidez} implementando en
    \textcolor{red}{un lenguaje compilado} y \textcolor{red}{mediante
      tipos estáticos}.
  \end{itemize}
\end{frame}

\begin{frame}[fragile]{Peculiaridades}
  \frametitle{Peculiariades}
  \begin{itemize}[<+->]
  \item Los ficheros tendrán \textcolor{red}{extensión \texttt{.spyx}}
    en lugar de \texttt{.sage}.
  \item \textcolor{red}{Serán escritos} en una \textcolor{red}{versión
      compilada} de \texttt{Python} llamada \texttt{Cython} (extensión
    de \texttt{Python} y \texttt{C}).
  \item El código en \textcolor{red}{\texttt{Cython} admite} la mayoría de rasgos de
    \texttt{Python}: listas por compresión, expresiones condicionales,
    código con \texttt{+=}, código de módulos de \texttt{Python}, etc.
  \item El código en \textcolor{red}{\texttt{Cython} admite} la
    mayoría de rasgos de \texttt{C}: declaración de variables, hacer
    llamadas a bibliotecas de \texttt{C}, etc.
  \item El código es \textcolor{red}{convertido} a código \texttt{C} y
    es \textcolor{red}{compilado} con un compilador de \texttt{C}.
  \end{itemize}
\end{frame}

\begin{frame}[fragile]{Peculiaridades}
  \frametitle{Peculiaridades}
  \begin{itemize}[<+->]
  \item La compilación ocurre \textcolor{red}{detrás del escenario},
    sin acción específica del usuario.
  \item Los archivos resultantes son \textcolor{red}{eliminados} al
    salir de \texttt{Sagemath}.
  \item Puede ser \textcolor{red}{invocado} \texttt{Sagemath} \textcolor{red}{desde} un
    esquema de \textcolor{red}{\texttt{Python}}, siempre que el directorio base de
    \texttt{Sagemath} este en el \texttt{PATH}.
  \item También puede ser \textcolor{red}{invocado} \texttt{Sagemath}
    \textcolor{red}{desde} un fichero escrito de acuerdo a las reglas
    de \textcolor{red}{\texttt{\LaTeX}}.
  \end{itemize}
\end{frame}

\section{Instalación}

\begin{frame}[fragile]{Instalación}
  \frametitle{Instalación}
  \begin{block}{Clave: actualmente es posible usar \texttt{Sagemath}
      en las
      \htmladdnormallink{platamormas}{https://www.sagemath.org/download.html}:}
    Mac OS, Linux, Solaris y OpenSolaris. A esa lista se añadió
    recientemente Windows.
  \end{block}\pause 
  \begin{itemize}[<+->]
  \item En \textcolor{red}{Mac OS} la instalación es según el proceso
    habitual a partir de un fichero \texttt{.dmg}, bien como una \textcolor{red}{app
    tradicional} o para ser \textcolor{red}{usado en la línea} de órdenes Unix.
  \item En \textcolor{red}{Linux} es por respositorio
    \htmladdnormallink{\textcolor{red}{PPA}}{https://launchpad.net/~aims/+archive/ubuntu/sagemath}
    o
    \htmladdnormallink{\textcolor{red}{manual}}{https://wildunix.es/posts/instalar-sage-en-mac-os-x-ubuntu-y-otros-linux-uso-bajo-windows/}.
  \item En Windows \htmladdnormallink{\textcolor{red}{descargaremos un
      .exe}}{https://github.com/sagemath/sage-windows/releases} y
    haremos una instalción al uso con él.
  \end{itemize}
\end{frame}

\begin{frame}[fragile]{Uso online}
  \frametitle{Uso online}
  \begin{block}{Clave: \texttt{Sagemath} puede ser usado online}
    siendo  \htmladdnormallink{cocalc}{https://cocalc.com/} el sitio
    más popular.
  \end{block}
\end{frame}

\section{De \texttt{sagenb} a \texttt{jupyter}}

\begin{frame}[fragile]{Reciclado}
  \frametitle{Reciclado}
  \begin{block}{Clave: el servidor \texttt{sagenb} está en absoluto
      declive por muchas razones, entre otras por su dificultad para
      interactuar con la nube.}
    Actualmente los usuarios se han volcado en el uso bajo
    \texttt{jupyter}. Si tenemos trabajo en ficheros \texttt{.sws}
    habremos de pasarlos a formato \texttt{.ipynb}.
  \end{block}\pause 
  \begin{itemize}[<+->]
  \item La operación se realiza sólo una vez con un esquema en
    \texttt{Python}
  \item El script es \texttt{sagenb-export} que se instala con
    \texttt{pip}.
  \item La recuperación es fichero a fichero y está detallada en
    \htmladdnormallink{wildunix}{https://wildunix.es/posts/instalar-sage-en-mac-os-x-ubuntu-y-otros-linux-uso-bajo-windows/}.
  \end{itemize}
\end{frame}

\section{Conclusiones}

\begin{frame}[fragile]{Conclusiones}
  \frametitle{Conclusiones}
  \begin{block}{Clave: Valoración de \texttt{Sagemath}}
      como sistema de computación.
  \end{block}\pause
  \begin{itemize}[<+->]
  \item Tener licencia \textcolor{OliveGreen}{GNU V2 o posterior}
  \item Usar \textcolor{OliveGreen}{Python} como lenguaje de programación.
  \item Haber sido capaz de dar servicio vía \textcolor{OliveGreen}{Jupyter}.
  \item La interesante solución de \textcolor{OliveGreen}{Cython} para
    beneficiar la rapidez.
  \item \textcolor{OliveGreen}{Aprovechar y Unificar} los distintos sistemas
    de cálculo que han dado resultado y están especializados en áreas
    del mismo.
  \item \textcolor{OliveGreen}{Bibliotecas propias} desarrolladas por el
    proyecto.
  \item Funcionamiento \textcolor{OliveGreen}{muy competitivo} en
    determinadas facetas.
  \end{itemize}
\end{frame}

\begin{frame}[fragile]{Conclusiones}
  \frametitle{Conclusiones}
  \begin{itemize}[<+->]
  \item Usar \textcolor{red}{Python 2.7} todavía.
  \item Hacer \textcolor{red}{hincapié} en \textcolor{red}{compatibilizar}.
  \item \textcolor{red}{Escasos recursos} para el desarrollo.
  \item \textcolor{red}{No} existir una \textcolor{red}{perspectiva
      clara} de \textcolor{red}{desarrollo de bibliotecas propias}.
  \item \textcolor{red}{No} ser más \textcolor{red}{accesible} el
    comité de desarrollo.
  \end{itemize}
\end{frame}

\end{document}

\begin{frame}[fragile]{}
  \frametitle{}
  \begin{block}{Clave: }
  \end{block}\pause 
  \begin{itemize}[<+->]
  \item
  \end{itemize}
\end{frame}

\begin{frame}[fragile]{}
  \frametitle{}
  \begin{itemize}[<+->]

  \end{itemize}
\end{frame}

\htmladdnormallink{palabra}{dirección}

